%% ----------------------------------------
%% Preamble
%% ----------------------------------------
\documentclass[]{scrartcl}

\usepackage[english, ngerman]{babel}			% Deutsche typogr. Regeln + Trenntabelle
\usepackage[babel, german=quotes]{csquotes}		% Anführungszeichen
\usepackage{graphicx}							% Tabellen
\usepackage{listings}							% Code Listings
\usepackage{xcolor}
\usepackage{enumerate}

%% ----------------------------------------
%% C-Code Listing style
%% ----------------------------------------
\definecolor{mGreen}{rgb}{0,0.6,0}
\definecolor{mGray}{rgb}{0.5,0.5,0.5}
\definecolor{mPurple}{rgb}{0.58,0,0.82}
\definecolor{backgroundColour}{rgb}{0.95,0.95,0.92}

\lstdefinestyle{CStyle}{
	backgroundcolor=\color{backgroundColour},   
	commentstyle=\color{mGreen},
	keywordstyle=\color{magenta},
	numberstyle=\tiny\color{mGray},
	stringstyle=\color{mPurple},
	basicstyle=\ttfamily\footnotesize,
	breakatwhitespace=false,         
	breaklines=true,                 
	captionpos=b,                    
	keepspaces=true,                 
	numbers=left,                    
	numbersep=5pt,                  
	showspaces=false,                
	showstringspaces=false,
	showtabs=false,                  
	tabsize=2,
	language=C
}


%% ----------------------------------------
%% Metadata
%% ----------------------------------------
\title{Zusatzübung Informatik}
\author{Thomas Wagner (\texttt{thomas.wagner@pa-systems.de})}


%% ----------------------------------------
%% Document
%% ----------------------------------------
\begin{document}
\lstset{style=CStyle}

\maketitle

\section{Zahlensysteme}

\subsection{Umrechnung}
Rechne die Zahlen in der Tabelle jeweils in alle Zahlensysteme um.

\begin{table}[h!]
	\centering
	\begin{tabular}{l|l|l|llll}
		\multicolumn{1}{c|}{\textbf{Dezimal/base10}} &
		\multicolumn{1}{c|}{\textbf{Binär/base2}} &
		\multicolumn{1}{c|}{\textbf{Oktal/base8}} &
		\multicolumn{1}{c}{\textbf{Hexadezimal/base16}} &
		\multicolumn{1}{c}{\textbf{}} &
		\multicolumn{1}{c}{\textbf{}} &
		\multicolumn{1}{c}{\textbf{}} \\ \hline
		1110  & 10001010110      & 2126   & 456  &  &  &  \\ \hline
		1023  & 1111111111       & 1777   & 3FF  &  &  &  \\ \hline
		44203 & 1010110010101011 & 126253 & ACAB &  &  &  \\ \hline
		231   & 11100111         & 347    & E7   &  &  &  \\ \hline
		123   & 01111011         & 173    & 7B   &  &  &  \\ \hline
		450   & 111000010        & 7C2    & 1C2  &  &  &  \\ \hline
		2706  & 10111000110      & 1178   & 5C6  &  &  &  \\ \hline
		1010  & 1111110010       & 1762   & 3F2  &  &  &  \\ \hline
		4110  & 1000000010000    & 1002   & 1010 &  &  &  \\ \hline
		1958  & 11110100110      & 3646   & 7A6  &  &  &
	\end{tabular}
\end{table}

\section{Variablen}

\subsection{Gleitkomma Berechnungen}
Welche Werte enthalten die nachfolgenden Variablen?

\begin{enumerate}[{1)}]
	\item \lstinline|int i = 12.6 * 2.0;|
	\item \lstinline|int i = 2.6 * 3;|
	\item \lstinline|int i = 10;|\\
	\lstinline|double d = ((i++ * ((i + 1) % 3)) - 1) / 2.0;|
\end{enumerate}

\subsection{Logische und binäre Berechnungen}
Welche Werte nimmt z jeweils an?\\
\lstinline|int x = 16;|\\
\lstinline|int y = 0;|\\
\lstinline|int z;|

\begin{enumerate}[{1)}]
	\item \lstinline|z = x >> 2;|
	\item \lstinline|z = x & y;|
	\item \lstinline|z = (x && y) ? 10 : 20;|
	\item \lstinline{z = x | y}
	\item \lstinline{z = x || y}
	\item \lstinline|z = x % (y + 3)|
\end{enumerate}

\end{document}
