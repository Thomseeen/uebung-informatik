%% ----------------------------------------
%% Preamble
%% ----------------------------------------
\documentclass[]{scrartcl}

\usepackage[english, ngerman]{babel}			% Deutsche typogr. Regeln + Trenntabelle
\usepackage[babel, german=quotes]{csquotes}		% Anführungszeichen
\usepackage{graphicx}							% Tabellen
\usepackage{listings}							% Code Listings
\usepackage{xcolor}
\usepackage{enumerate}							% Auflistungen mit Custom Symbolen
\usepackage{verbatim}							% Multiline Kommentare
\usepackage[hidelinks]{hyperref}							% Hyperlinks

%% ----------------------------------------
%% C-Code Listing style
%% ----------------------------------------
\definecolor{mGreen}{rgb}{0,0.6,0}
\definecolor{mGray}{rgb}{0.5,0.5,0.5}
\definecolor{mPurple}{rgb}{0.58,0,0.82}
\definecolor{backgroundColour}{rgb}{0.95,0.95,0.92}

\lstdefinestyle{CStyle}{
	backgroundcolor=\color{backgroundColour},   
	commentstyle=\color{mGreen},
	keywordstyle=\color{magenta},
	numberstyle=\tiny\color{mGray},
	stringstyle=\color{mPurple},
	basicstyle=\ttfamily\footnotesize,
	breakatwhitespace=false,         
	breaklines=true,                 
	captionpos=b,                    
	keepspaces=true,                 
	numbers=left,                    
	numbersep=5pt,                  
	showspaces=false,                
	showstringspaces=false,
	showtabs=false,                  
	tabsize=2,
	language=C
}


%% ----------------------------------------
%% Metadata
%% ----------------------------------------
\title{Zusatzübung Informatik}
\author{Thomas Wagner (\href{mailto:thomas.wagner@pa-systems.de}{\texttt{thomas.wagner@pa-systems.de}})}

%% ----------------------------------------
%% Document
%% ----------------------------------------
\begin{document}
\lstset{style=CStyle}

\maketitle

\section{Zahlensysteme}

\subsection{Umrechnung}
Rechne die Zahlen in der Tabelle jeweils in alle Zahlensysteme um.

\begin{table}[h!]
	\centering
	\begin{tabular}{l|l|l|llll}
		\multicolumn{1}{c|}{\textbf{Dezimal/base10}} &
		\multicolumn{1}{c|}{\textbf{Binär/base2}} &
		\multicolumn{1}{c|}{\textbf{Oktal/base8}} &
		\multicolumn{1}{c}{\textbf{Hexadezimal/base16}} &
		\multicolumn{1}{c}{\textbf{}} &
		\multicolumn{1}{c}{\textbf{}} &
		\multicolumn{1}{c}{\textbf{}} \\ \hline
		%1110  & 10001010110      & 2126   & 456  &  &  &  \\ \hline
		%1023  & 1111111111       & 1777   & 3FF  &  &  &  \\ \hline
		%44203 & 1010110010101011 & 126253 & ACAB &  &  &  \\ \hline
		%231   & 11100111         & 347    & E7   &  &  &  \\ \hline
		%123   & 01111011         & 173    & 7B   &  &  &  \\ \hline
		%450   & 111000010        & 7C2    & 1C2  &  &  &  \\ \hline
		%2706  & 10111000110      & 1178   & 5C6  &  &  &  \\ \hline
		%1010  & 1111110010       & 1762   & 3F2  &  &  &  \\ \hline
		%4110  & 1000000010000    & 1002   & 1010 &  &  &  \\ \hline
		%1958  & 11110100110      & 3646   & 7A6  &  &  &
		1110 &             &      &      &  &  &  \\ \hline
		     &             &      & 3FF  &  &  &  \\ \hline
		  	 &             &      & ACAB &  &  &  \\ \hline
		     & 11100111    &      &      &  &  &  \\ \hline
		     &             & 173  &      &  &  &  \\ \hline
		450  &             &      &      &  &  &  \\ \hline
		     & 10111000110 &      &      &  &  &  \\ \hline
		     &             & 1762 &      &  &  &  \\ \hline
		     &             & 1002 &      &  &  &  \\ \hline
		     &             &      & 7A6  &  &  &
	\end{tabular}
\end{table}

\section{Variablen}

\subsection{Gleitkomma Berechnungen}
Welche Werte enthalten die nachfolgenden Variablen?

\begin{enumerate}[{1)}]
	\item \lstinline|int i = 12.6 * 2.0;|
	\item \lstinline|int j = 2.6 * 3;|
	\item \lstinline|int k = 10;|\\
	\lstinline|double d = ((i++ * ((i + 1) % 3)) - 1) / 2.0;|
\end{enumerate}

\subsection{Logische und binäre Berechnungen}
Welche Werte nimmt z jeweils an?\\
\lstinline|int x = 16;|\\
\lstinline|int y = 0;|\\
\lstinline|int z;|

\begin{enumerate}[{1)}]
	\item \lstinline|z = x >> 2;|
	\item \lstinline|z = x & y;|
	\item \lstinline|z = (x && y) ? 10 : 20;|
	\item \lstinline{z = x | y;}
	\item \lstinline{z = x || y;}
	\item \lstinline|z = x % (y + 3);|
\end{enumerate}

\subsection{Gemischte Typen}
Welche Werte haben die angegebenen Variablen?\\

\begin{enumerate}[{1)}]
	\item \lstinline|int x = 25 / 4;|
	\item \lstinline|double y = 19.0 / 2;|
	\item \lstinline|int diff = 'D' - 'B';|
	\item \lstinline|int x = 5;|\\
	\lstinline|float f = ++x / 3.0;|
	\item \lstinline|char c = 0xB0;|\\
	\lstinline|double d = (c++ % 176) * 3.141;|
\end{enumerate}

\section{Programme}
Schreibe je ein vollständiges C-Programm, dass die folgende Funktionalität besitzt.
\begin{enumerate}[{1)}]
	\item Vom Benutzer werden zwei Zahlen eingelesen. Es wird die größere und die kleinere Zahl ausgegeben, sowie der Mittelwert.
	
	\item Vom Benutzer werden drei Zahlen eingelesen. Sie werden der Größe nach geordnet ausgegeben.
	
	\item Es werden alle Zahlen von 100 bis 200 ausgegeben, die weder durch 3, noch durch 7 teilbar sind.
	
	\item Erzeuge mithilfe einer Schleife eine Tabelle in der Konsole, die ASCII-Symbole mit ihrem jeweiligen Byte-Wert darstellt. Beispiel:\\
	\texttt{
		65 - A - a\\
		66 - B - b\\
		67 - C - c\\
		...\\
		90 - Z - z\\
	}

	\item Vom Benutzer wird eine Zahl eingelesen. Die Quersumme der Zahl wird ausgegeben.
	
	\item \textbf{Caesar-Verschlüsselung}: Vom Benutzer wird eine Zahl (Übersetzungsschlüssel) und ein Zeichen eingelesen. Das Zeichen wird um den Wert des Übersetzungsschlüssels im Alphabet verschoben und ausgegeben. Das Zeichen soll nur ein Buchstabe sein, bei falscher Eingabe soll eine Fehlermeldung ausgegeben werden. Beachte Klein- und Großschreibung! Beispiel:\\
	\texttt{
		Schlüssel: 3, Zeichen: d => g\\
		Schlüssel: 5, Zeichen: Y => D\\
		Schlüssel: 5, Zeichen: \& => Error
	}

	\item \textbf{Reziproke Summe}: Es werden vom Benutzer Zahlen eingelesen und die reziproke Summe berechnet: $y = \frac{1}{x_1} + \frac{1}{x_2} + \frac{1}{x_3} + \ldots + \frac{1}{x_n}$. Das Zwischenergebnis wird nach jeder Eingabe ausgegeben. Das Programm wird beendet, sobald der Benutzer 0 eingibt. Dann wird das Endergebnis ausgegeben.
	
	\item \textbf{Einstellige Quersumme}: Vom Benutzer wird eine Zahl eingelesen. Die Quersumme wird berechnet. Es wird solange die Quersumme des jeweiligen Ergebnisses berechnet, bis eine einstellige Zahl rauskommt. Alle Zwischenschritte werden ausgegeben. Zusatz: Die Berechnung einer Quersumme wird in eine Funktion ausgelagert. Beispiel:
	\texttt{
		839568 => 39 => 12 => 3
	}

	\item Die ersten 20 Glieder der Fibonacci-Zahlen werden in ein Array geschrieben und ausgegeben. Definition für die Reihe:\\
	\texttt{
		F[0] = 0\\
		F[1] = 1\\
		F[n] = F[n-1] + F[n-2]\\
	}

	\item Vom Benutzer wird eine römische Zahl eingelesen. Diese wird als arabische Zahl wieder ausgegeben. Tipp:
	\texttt{
		$I \widehat{=}    1$,
		$V \widehat{=}    5$,
		$X \widehat{=}   10$,
		$L \widehat{=}   50$,
		$C \widehat{=}  100$,
		$D \widehat{=}  500$,
		$M \widehat{=} 1000$
	}

	\item Vom Benutzer wird eine Zahl eingelesen. Diese wir als römische Zahl wieder ausgegeben.
\end{enumerate}

\section{Funktionen}
Schreibe je eine Funktion, die die folgenden Funktionalität besitzt. Überlege, welche Parameter und welcher Rückgabewert sinnvoll sind.

\begin{enumerate}[{1)}]
	\item Berechne die Fakultät einer Zahl. Schreibe hierzu eine rekursive und eine iterative Variante.
	
	\item Prüfe ob eine Zahl gerade oder ungerade ist.
	
	\item Ermittle die Anzahl der Großbuchstaben in einer Zeichenkette.
	
	\item Ermittle den Durchschnitt aller Werte eines Integer-Arrays.
	
	\item Ermittle das Maximum aller Werte eines Integer-Arrays.
	
	\item Ermittle den Funktionswert der Funktion $f(x) = \frac{1}{x^2 - 3}$ an der Stelle $x$.
	
	\item (Extra) Prüfe ob eine Zahl durch 6 teilbar ist, ohne den Modulo-Operator zu verwenden.
	
	\item (Extra) Ermittle den Median aller Werte eines Integer-Arrays. Tipp: Der Median der Zahlen 2,3,4,8 ist 4. Der Median der Zahlen 2,3,5,8 ist 3.5.
\end{enumerate}

\section{Sonstiges}
\begin{enumerate}[{1)}]
\item Finde alle Fehler im Programm (Syntax und Logik).
\begin{lstlisting}
#define CNT 10

int main() {
	int arr[CNT], jj;
	
	for (ii = 0; ii < CNT; ii++) {
		printf("Enter a number: ");
		scanf("%ii", arr[ii]);
	}

	for (jj = 0; jj < CNT; jj++)	{
		arr[jj] =+ arr[jj+1];
		printf("Number %i is %i\n", jj, arr[jj]);
	}

	return 0
}
\end{lstlisting}

\item Was macht das folgende Programm?
\begin{lstlisting}
#include <stdio.h>

int main() {
	int a,b,c;
	
	printf("Enter three numbers: ");
	scanf("%d", &a);
	scanf("%d", &b);
	scanf("%d", &c);
	
	int wtf = a < b ? (c < a ? a : (c < b ? c : b)) : (c < b ? b : (c < a ? c : a));
	
	printf("The result is: %d\n", wtf);
	
	return 0;
}
\end{lstlisting}

\item Was berechnet die folgende Funktion?
\begin{lstlisting}	
unsigned long long magic(unsigned int y) {
	return y == 0 ? 1 : 2ull << (y - 1);
}
\end{lstlisting}

\item Was macht das folgende Programm?
\begin{lstlisting}
#include <stdio.h>

int main() {
	char aa[100], bb[100];
	printf("\nEnter the first string: ");
	scanf("%s", aa);
	printf("\nEnter the second string to be concatenated: ");
	scanf("%s", bb);

	char *a = aa;
	char *b = bb;

	while(*a) {
		a++;
	}

	while(*b) {
		*a = *b;
		b++;
		a++;
	}

	*a = '\0';

	printf("\n\n\nThe string after concatenation is: %s", aa);
	return 0;
}
\end{lstlisting}
\end{enumerate}

\section{Links}
\begin{enumerate}[{1)}]
	\item \href{https://github.com/Thomseeen/uebung-informatik}{https://github.com/Thomseeen/uebung-informatik} (Repository für dieses Dokument)
\end{enumerate}

\begin{comment}
\subsection{CAN-Frames}
Ein CAN-Frame hat 8 Byte Daten. Oft werden in diese 8 Byte mehrere Messwerte codiert. Eine Beschreibungsdatei gibt dann an, an welcher Bit-Position sich ein Wert befindet und welche Bitlänge er hat. So können aus den 8 Byte Daten die für den jeweiligen Messwert relevanten Bits genommen werden. Die sich ergebende Zahl muss dann für den eigentlichen Messwert noch mit einem Faktor und einem Offset verrechnet werden.\\

Implementiere die passende Funktion \lstinline{float GetValueFromFrame(byte[] data, int pos, int offset, float val_factor, float val_offset);}

\begin{enumerate}[{1)}]
	\item Erstelle den Funktionskörper
	\item Kopiere aus dem Byte-Array die relevanten Bytes in ein zweites Byte Array
	\item Schiebe die Bytes aus dem Array in ein \lstinline|long|, um eine gültige Zahleninterpretation zu erhalten
	\item Maskiere und schiebe das \lstinline|long|, um die genauen Bitpositionen/-länge einzuhalten
	\item Verrechne das Ergebnis mit dem Faktor und Offset
\end{enumerate}
\end{comment}

\end{document}
