%% ----------------------------------------
%% Preamble
%% ----------------------------------------
\documentclass[]{scrartcl}

\usepackage[english, ngerman]{babel}			% Deutsche typogr. Regeln + Trenntabelle
\usepackage[babel, german=quotes]{csquotes}		% Anführungszeichen
\usepackage{graphicx}							% Tabellen
\usepackage{listings}							% Code Listings
\usepackage{xcolor}
\usepackage{enumerate}

%% ----------------------------------------
%% C-Code Listing style
%% ----------------------------------------
\definecolor{mGreen}{rgb}{0,0.6,0}
\definecolor{mGray}{rgb}{0.5,0.5,0.5}
\definecolor{mPurple}{rgb}{0.58,0,0.82}
\definecolor{backgroundColour}{rgb}{0.95,0.95,0.92}

\lstdefinestyle{CStyle}{
	backgroundcolor=\color{backgroundColour},   
	commentstyle=\color{mGreen},
	keywordstyle=\color{magenta},
	numberstyle=\tiny\color{mGray},
	stringstyle=\color{mPurple},
	basicstyle=\ttfamily\footnotesize,
	breakatwhitespace=false,         
	breaklines=true,                 
	captionpos=b,                    
	keepspaces=true,                 
	numbers=left,                    
	numbersep=5pt,                  
	showspaces=false,                
	showstringspaces=false,
	showtabs=false,                  
	tabsize=2,
	language=C
}


%% ----------------------------------------
%% Metadata
%% ----------------------------------------
\title{Zusatzübung Informatik}
\author{Thomas Wagner (\texttt{thomas.wagner@pa-systems.de})}


%% ----------------------------------------
%% Document
%% ----------------------------------------
\begin{document}
\lstset{style=CStyle}

\maketitle

\section{Zahlensysteme}

\subsection{Umrechnung}
Rechne die Zahlen in der Tabelle jeweils in alle Zahlensysteme um.

\begin{table}[h!]
	\centering
	\begin{tabular}{l|l|l|llll}
		\multicolumn{1}{c|}{\textbf{Dezimal/base10}} &
		\multicolumn{1}{c|}{\textbf{Binär/base2}} &
		\multicolumn{1}{c|}{\textbf{Oktal/base8}} &
		\multicolumn{1}{c}{\textbf{Hexadezimal/base16}} &
		\multicolumn{1}{c}{\textbf{}} &
		\multicolumn{1}{c}{\textbf{}} &
		\multicolumn{1}{c}{\textbf{}} \\ \hline
		%1110  & 10001010110      & 2126   & 456  &  &  &  \\ \hline
		%1023  & 1111111111       & 1777   & 3FF  &  &  &  \\ \hline
		%44203 & 1010110010101011 & 126253 & ACAB &  &  &  \\ \hline
		%231   & 11100111         & 347    & E7   &  &  &  \\ \hline
		%123   & 01111011         & 173    & 7B   &  &  &  \\ \hline
		%450   & 111000010        & 7C2    & 1C2  &  &  &  \\ \hline
		%2706  & 10111000110      & 1178   & 5C6  &  &  &  \\ \hline
		%1010  & 1111110010       & 1762   & 3F2  &  &  &  \\ \hline
		%4110  & 1000000010000    & 1002   & 1010 &  &  &  \\ \hline
		%1958  & 11110100110      & 3646   & 7A6  &  &  &
		1110 &             &      &      &  &  &  \\ \hline
		     &             &      & 3FF  &  &  &  \\ \hline
		  	 &             &      & ACAB &  &  &  \\ \hline
		     & 11100111    &      &      &  &  &  \\ \hline
		     &             & 173  &      &  &  &  \\ \hline
		450  &             &      &      &  &  &  \\ \hline
		     & 10111000110 &      &      &  &  &  \\ \hline
		     &             & 1762 &      &  &  &  \\ \hline
		     &             & 1002 &      &  &  &  \\ \hline
		     &             &      & 7A6  &  &  &
	\end{tabular}
\end{table}

\section{Variablen}

\subsection{Gleitkomma Berechnungen}
Welche Werte enthalten die nachfolgenden Variablen?

\begin{enumerate}[{1)}]
	\item \lstinline|int i = 12.6 * 2.0;|
	\item \lstinline|int j = 2.6 * 3;|
	\item \lstinline|int k = 10;|\\
	\lstinline|double d = ((i++ * ((i + 1) % 3)) - 1) / 2.0;|
\end{enumerate}

\subsection{Logische und binäre Berechnungen}
Welche Werte nimmt z jeweils an?\\
\lstinline|int x = 16;|\\
\lstinline|int y = 0;|\\
\lstinline|int z;|

\begin{enumerate}[{1)}]
	\item \lstinline|z = x >> 2;|
	\item \lstinline|z = x & y;|
	\item \lstinline|z = (x && y) ? 10 : 20;|
	\item \lstinline{z = x | y;}
	\item \lstinline{z = x || y;}
	\item \lstinline|z = x % (y + 3);|
\end{enumerate}

\subsection{Gemischte Typen}
Welche Werte haben die angegebenen Variablen?\\

\begin{enumerate}[{1)}]
	\item \lstinline|int x = 25 / 4;|
	\item \lstinline|double y = 19.0 / 2;|
	\item \lstinline|int diff = 'D' - 'B';|
	\item \lstinline|int x = 5;|\\
	\lstinline|float f = ++x / 3.0;|
	\item \lstinline|char c = 0xB0;|\\
	\lstinline|double d = (c++ % 176) * 3.141;|
\end{enumerate}

\section{Programme}
\subsection{CAN-Frames}
Ein CAN-Frame hat 8 Byte Daten. Oft werden in diese 8 Byte mehrere Messwerte codiert. Eine Beschreibungsdatei gibt dann an, an welcher Bit-Position sich ein Wert befindet und welche Bitlänge er hat. So können aus den 8 Byte Daten die für den jeweiligen Messwert relevanten Bits genommen werden. Die sich ergebende Zahl muss dann für den eigentlichen Messwert noch mit einem Faktor und einem Offset verrechnet werden.\\

Implementiere die passende Funktion \lstinline{float GetValueFromFrame(byte[] data, int pos, int offset, float val_factor, float val_offset);}

\begin{enumerate}[{1)}]
	\item Erstelle den Funktionskörper
	\item Kopiere aus dem Byte-Array die relevanten Bytes in ein zweites Byte Array
	\item Schiebe die Bytes aus dem Array in ein \lstinline|long|, um eine gültige Zahleninterpretation zu erhalten
	\item Maskiere und schiebe das \lstinline|long|, um die genauen Bitpositionen/-länge einzuhalten
	\item Verrechne das Ergebnis mit dem Faktor und Offset
\end{enumerate}

\end{document}
